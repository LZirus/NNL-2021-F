
\section{Project - Description}

This project was conducted in context of a university course at Université Côte
d'Azur in Nice, France. The main goal of the project was a first experimentation
with the concept of machine learning with neural networks. It's purpose was to
gain a deeper understanding of what influences the decisions of such a model and
using the gained knowledge to create a simple, but fully functioning
face-mask-detector.
\newline
The project was split into two parts, the data collection and preparation and
the machine learning (ML) model creation and training. In the following both
parts will be outlined with a introduction of the team members and their roles
in both parts.

\subsection{Part I - Data Collection and Preparation}

The data collection is already a big part of the project, as there are a lot of
things that can go wrong here. The data is the main thing that influences the
final model in the decision it makes. A ML model is in this context not
different than humans, as it can only act on its own experience. Therefore
making sure the data is chosen in conjunction with the final goals is essential.
As the scale of this project is fairly small and its purpose is a first
experimentation with the concept, it was sufficient to expect the the model to
recognize the three members of this project.
\newline
The collected data will most likely not be in a format, that is easily readable
for our model. The next step therefore needs to manually interpret the data,
label it for the model and standardize it. The level of the data preprocessing
again is determined by the goal one wants to achieve. In the scale of this
project, our preprocessing will go as far as already identifying the face on the
image, cutting it to a good fitting square and giving it a label according to
its context.
\newline
To achieve all that in a streamlined, standardized and user friendly way, the
development of an image annotation software was necessary. In addition to the
manual annotation there are some automation tricks that can be used to help with
easy classifiable data. 

\subsection{Part II - Machine Learning Model}

This of course is the center of the project. Setting up and compiling a ML
model, that is capable of taking an image of a face and determining whether the
person shown is wearing a mask and what kind if yes.
\newline
As mentioned before, the goal was limited to being able to detect masks on the
faces of the team members of this project, as we would have needed a much larger
and more diverse dataset to achieve reliable results with all other faces. Still
the hope is, that the model will be able to identify more faces, that have
similarities to the team members. This will most likely include white, western
men, while there will be difficulties with different skin colors and facial
structures.

\subsection{Team Members and Roles}

The team consists of three members, Michael Cegielka, Julien Schulz and Leonhard
Zirus. The roles in this project where distributed equally in a way where
everyone was still involved in every part of it. 
\newline
In the first part of the project, Leonhard took charge of the annotation
software, being responsible for the organization and collaboration. He created
the framework and interfaces. Julien as our ML expert was responsible for making
sure that the software was usable in the second part of the project and created
a streamlined process for the creation of the dataset, augmentation and later
labeling. Micheal was mainly in charge of the UI (User Interface) and UE (User
Experience) of the annotation software. 
\newline
TODO: Part II 
TODO: Who coded
TODO: Who wrote the paper