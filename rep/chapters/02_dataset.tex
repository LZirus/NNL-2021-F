\section{Dataset}

The used dataset was created by taking pictures of our team members wearing
different kinds of masks. For the ML model to use the images they need to be
preprocessed and labeled to enable the learning process.

\subsection{Image Creation}

The images taken will influence greatly what kind of images the model will later
be able to recognize. This means, that lighting, orientation of the face,
background, clothing or other kinds of patterns are very important to be aware
of when creating the pictures. 
\newline
Imagine a person choses to never wear a bonnet when walking inside, but because
of corona, that is exactly when he/she always wears a mask. As it is winter, the
opposite is true for being outside, he/she wears a bonnet, as it is cold, but no
mask, because it is outside and not required. Now using pictures from these
situations, the model might turn to identifying whether the person is wearing a
bonnet instead, as the dataset reflects such an implication. This is just one of
many examples, of unwilling implications that might turn up in a dataset.
\newline
In order to make sure to not have any such implications the images were tried to
create, using equal distributions of different backgrounds, with no lighting
differences correlating to the wearing or not wearing of a mask and different
clothing styles not related to the mask. The later one was limited to wearing a
bonnet or not.
\newline
The limitation of time and equipment will surely create a few problems with the
dataset. As an example, the creation started in the evening, which in turn
changed the lighting between changing to a different mask. As this problem
became aware, the fotoshoot was moved inside to at least have similar lighting
throughout all the pictures.
\newline
The improvements to the creation are countless, but would have all not been
justified by the dimension of this project. To create a proper one, the
diversity of the people, lighting, backgrounds, clothing, camera-lenses, etc.
would all have to be greatly increased. Also the number of images created is
still very small for the model to accurately recognize anyone, anywhere we put
them in front of the camera. In the end the created dataset should have
sufficient diversity to serve the purpose of this project.

\subsection{Scaling, Labeling and Formatting}

With the base for the dataset created, it is time to pre-process the images into
a format that can be used for the model. The final images should have a square
format with 240 x 240 pixels saved in a folder specifying it's label. There are
four labels: "no\_mask", "ffp2", "op\_mask" and "other\_mask". In this case
"other\_mask" will be different kinds of cloth-masks.
\newline
To limit the manual work as much as possible, the usage of the OpenCV
Haar-Cascade-Classifier, which is a pre-trained algorithm to detect faces on an
image. A short script enabled the automated loading of all images in a folder,
computation of present faces, cropping of the face in the image and saving it in
a new location. As the cascade classifier gives a whole lot of false positives,
it is important to got through the exported files and delete anything that can
not be used. To help figure out which image was already correctly processed,
there is a second small script comparing the processed images to the raw images
and copying all unprocessed ones to a new folder, ready to be manually
annotated. 
\newline
The labeling process of course can't be automated, as this would already be the
finished model. The processed images therefore have to be manually copied in
their respective folders.
\newline
The missing images, now have to be manually annotated, where finally the
image annotator comes to play. Using this software it is just a matter of going
through all remaining images, marking each face with a bounding box and labeling
it correctly. Afterwards the Annotator offers a crop and save function, that
takes care of cropping, scaling and correctly saving the images in the right
labeled folders.
