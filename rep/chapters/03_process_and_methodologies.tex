\section{Implementation and Methodologies}

This project had two major implementations, the annotator and the ML model,
which will be detailed in this chapter. The following will explain used
methodologies and choices made in the implementation process as well as overcome
difficulties. 

\subsection{Image-Annotator}

The purpose of the image annotation software is to give an easy possibility of
labeling, rescaling, cropping and saving images in order to be used as dataset
by the ML model.
\newline
The software was completely written in python, an interpreted script language.
Still python has object oriented features, that where partly used in this
project. 
\newline
The program program is structured in a way, to have two different classes,
"Window" and "popupWindow" to take care of the UI part of the project. The rest
of the functionality is structured into script-like functions. This structuring
makes it easier to split the program into smaller tasks that can be distributed
and worked on individually. This is a methodology oriented at the MVC (Model
View Controller) software development pattern. The UI is here clearly separated
from the controlling instance, that handles the backend of user interaction as
well as the model, which is responsible for data storage and maintenance. As
this is a simple single instance application, of course a detailed
implementation was not necessary and the model and controller are somewhat mixed
and both just represented as the separate functions. Still it gives the
possibility of separating the implementation of the UI, the backend and the
storage system.
\newline
The UI is handled by the the tkinter python library, which gives a lot of basic
functionality to easily implement the frontend of the application. The UI of
this project was mainly focused on the menu bar on top in addition to a right
click menu. This choice was made in order to keep the overall frame clean and
focused on the most important thing, the image.
\newline
All created data


\subsection{Machine Learning Model}