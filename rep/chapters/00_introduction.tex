\section{Introduction}

With software infusing more and more of our daily lives, software developers are
trying to keep up with the rising expectations of end-users. With IoT in the
rise, we collect more data in a single day, than we could have ever imagined and
have digital interfaces to nearly all parts of our lives. 


Software development has grown to a point, where we have a giant base of great
algorithms, that enable us to solve many problems quickly and efficiently. In
todays software development it often comes down to having an idea and
implementing it on the basis of these existing algorithms. Still there is a
limit to what they can do and it is often reached at tasks that seam pretty
simple to us humans. Algorithms are great in using big amounts of data,
accumulating it to models and trends, showing it in smart graphs and in general
making it visual and easier to analyse and interpret. Still the final
interpretation is a skill where humans are unmatched in. 


The following paper follows the entire process of creating a so called
"artificial intelligence" that is able to take an image, grabbed from a file or
a video and determine, if showing a person, which kind of mask this person is
wearing or whether he/she is wearing any at all. First the project and its goals
will be layed out in detail with a short introduction of each team member and
their role in this project. This is followed by a description of the used
dataset and how it was created. Then there will be an outline of the process
including used methodologies and encountered difficulties and finally a
summarization of the obtained results and their interpretation. 