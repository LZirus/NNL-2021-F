\section{Introduction}

Software development has grown to a point, where we have a giant base of great
algorithms, that enable us to solve many problems quickly and efficiently. Nowadays
it often comes down to having an idea and implementing it on the basis of these
existing algorithms. Still there is a
limit to what they can do and it is often reached at tasks that seem pretty
simple to us humans. Algorithms are great at using big amounts of data,
accumulating it to models and trends, showing it in smart graphs, generating
visual representations and simplifying the analysis and interpretation. Still
the final interpretation is a skill humans have yet to be matched in.
\newline
These days, more and more companies are starting to implement machine learning
(ML) algorithms as support to their existing algorithms. These ML models are
the closest a program can come to the human ability of using experience and learning
from it, in order to interpret new experiences on the basis of old ones.
While they have been around for a while now, they are still not reliable enough
to be fully trustworthy. As the topic becomes more and more important in
todays world, getting to know it, can prove very valuable in the future.
\newline
This paper follows the entire process of creating a so called "artificial
intelligence" that is able to take an image, taken from a file or a video and
determine, if showing a person, which kind of mask this person is wearing or
whether he/she is wearing any at all. First the project and its goals will be
layed out in detail with a short introduction of each team member and their role
in this project. This is followed by a description of the used dataset and how
it was created. Then there will be an outline of the process including used
methodologies and encountered difficulties and finally a summarization of the
obtained results and their interpretation. 