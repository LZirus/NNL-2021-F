\section{Summary and Outlook}

In this project the entire process from creating a dataset until training and
using a neural network were shown. Even though the dataset was very small and
not professionally created, the produced neural network is able to detect many
different faces and can accurately determine wether a person is wearing no mask,
an op-mask or a ffp2-mask.
\newline
The provided tests show promising results, but of course there is still room for
improvement. Mentioned multiple times now, recreating the dataset would be the
first step in this direction. Especially if a goal would be for the program to
recognize any human on earth, there is a lot of training data missing. Right now
the program is most likely racist in a lot of different ways, as from it's
training data it is only used to the faces of the team members.
\newline
Another obvious step for the future would be running a proper tuner on the model
to better determine a lot of the hyperparameters. This includes mostly the
specific layer settings. Fine tuning the model can probably get the accuracy to
values consistently over 90%.
\newline
This project has nicely shown, how accessible the technology of machine learning
is nowadays and how easy it can be to implement an already very powerful model.
In our opinion machine learning and the development of "artificial intelligence"
will play a big role in the future. It has a lot of still unused potential.
With how easy it is to get started in the topic, building up a basic knowledge
in this area is recommended for every software engineer and even hobbyist.
\newline
We are very much looking forward to further experimentation and learning in
machine learning.